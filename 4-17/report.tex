\documentclass[a4paper,12pt]{article}
\usepackage{listings}
\title{「プログラミング言語」課題}
\author{1029-24-9540 山崎啓太郎}
\begin{document}
\lstset{numbers=left,basicstyle=\small}
\maketitle

\section{Ex. 3.1}
\subsection{考え方}
make-accumulatorで返される関数内で、make-acumulatorに与えられた引数に破壊的代入を使うことにより、それを更新していく。\\
3.1.scmではmake-accumulator関数の引数xに対して、
\verb\(set! x (+ x add))\とすることでxを更新している。\\
\subsection{実行例}
test/3.1.scm\\
\lstinputlisting{test/3.1.scm}
結果\\
{\small
\begin{verbatim}
Testing 3.1 ...                                                  
test (A 10), expects 15 ==> ok
test (A 10), expects 25 ==> ok
\end{verbatim}
}

\section{Ex. 3.3}
\subsection{考え方}
make-account関数で与えられたパスワードは、口座の操作ごとに入力する仕組みになっている。\\
3.3.scmでは口座の操作の際にdispatch関数が呼ばれているため、dispatch関数内で口座の操作をする前にパスワードが正しいかどうかを判断し、パスワードが正しければ操作をし、正しくなければ"Incorrect password"を返せばよい。\\
\subsection{実行例}
test/3.3.scmより\\
{\small
\begin{verbatim}
Testing 3.3 ...                                                  
<Account test>-----------------------------------------------------------------
test ((acc 'secret-password 'withdraw) 30), expects 70 ==> ok
test ((acc 'secret-password 'diposit) 40), expects 110 ==> ERROR: GOT #<<error> "Unknown request -- MAKE-ACCOUNT diposit">
test ((acc 'secret-password 'withdraw) 300), expects #<error> ==> ERROR: GOT "Insufficient funds"
test ((acc 'secret-password 'mes) 10), expects #<error> ==> ok
<Password test>----------------------------------------------------------------
test ((acc 'secret-password 'withdraw) 40), expects 60 ==> ok
test ((acc 'some-other-password 'diposit) 50), expects "Incorrect password" ==> ok
\end{verbatim}
}

\end{document}
