\documentclass[a4paper,12pt]{article}
\usepackage{listings}
\title{「プログラミング言語」課題}
\author{1029-24-9540 山崎啓太郎}
\begin{document}
\lstset{numbers=left,basicstyle=\small}
\maketitle

\section{Ex. 3.1}
\subsection{考え方}
make-accumulatorで返される関数内で、make-acumulatorに与えられた引数に破壊的代入を使うことにより、それを更新していく。\\
3.1.scmではmake-accumulator関数の引数xに対して、
\verb\(set! x (+ x add))\とすることでxを更新している。\\
\subsection{実行例}
test/3.1.scm\\
\lstinputlisting{test/3.1.scm}
結果\\
{\small
\begin{verbatim}
Testing 3.1 ...                                                  
test (A 10), expects 15 ==> ok
test (A 10), expects 25 ==> ok
\end{verbatim}
}

\section{Ex. 3.3}
\subsection{考え方}
make-account関数で与えられたパスワードは、口座の操作ごとに入力する仕組みになっている。\\
3.3.scmでは口座の操作の際にdispatch関数が呼ばれているため、dispatch関数内で口座の操作をする前にパスワードが正しいかどうかを判断し、パスワードが正しければ操作をし、正しくなければ"Incorrect password"を返せばよい。\\
\subsection{実行例}
test/3.3.scm\\
\lstinputlisting{test/3.3.scm}
結果\\
{\small
\begin{verbatim}
Testing 3.3 ...                                                  
<Account test>-----------------------------------------------------------------
test 30引出, expects 70 ==> ok
test 40振込, expects 110 ==> ok
test 300引出, expects "Insufficient funds" ==> ok
test 存在しない操作をする, expects #<error> ==> ok
<Password test>----------------------------------------------------------------
test 正しいパスワードで40引出, expects 60 ==> ok
test 間違ったパスワードで50振込, expects "Incorrect password" ==> ok
\end{verbatim}
}

\section{Ex. 3.7}
\subsection{考え方}
make-jointは第一引数にmake-accountで生成された口座(account)、第二引数にそのaccountのパスワード(old-password)、第三引数に新しく作るaccount用のパスワード(old-password)を取る。\\
まず、make-jointではold-passwordが正しいものか判定しなくてはいけないが、accountにはそのための機能が備わっていないため、make-accountを改良する必要がある。\\
3.7.scmでは\verb\(account old-password 'check-password)\を評価することで、old-passwordが正しい場合trueが、間違っていた場合falseが帰ってくるようになっている。\\
old-passwordが正しかった場合、新規accountで操作をするためのlambda式を返す。\\
そのlambda式はmake-accountで生成された口座と同様に、パスワードと操作のためのメッセージを引数としてとり、与えられたパスワードがnew-passwordと一致していればaccountにメッセージを渡す。\\
以上のようにすれば、make-jointで必要な機能を満たすことができる。\\
\subsection{実行例}
test/3.3.scm\\
\lstinputlisting{test/3.7.scm}
結果\\
{\small
\begin{verbatim}
Testing 3.7 ...                                                  
test peter-accを使って40引出, expects 60 ==> ok
test さらにpaul-accを使って40引出, expects 20 ==> ok
test さらにpaul-accを使って30振込, expects 50 ==> ok
test peter-accのパスワードを間違える, expects "Incorrect password" ==> ok
test paul-accのパスワードを間違える, expects "Incorrect password" ==> ok
\end{verbatim}
}

\end{document}
