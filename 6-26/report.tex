\documentclass[a4paper,12pt]{article}
\usepackage{listings}
\title{「プログラミング言語」課題}
\author{1029-24-9540 山崎啓太郎}
\begin{document}
\lstset{numbers=left,basicstyle=\small}
\maketitle

\section{Ex. 4.27}
\subsection{考え方}
式の値がサンクの時は、それがサンクでないものに達するまで強制を繰り返すため、wを定義した際に、外側のidが実行され、内側のidは実行されない。\\
そのため、一回目のcountの評価結果は1となる。\\
wを評価すると内側のidが実行され10が評価値となる。\\
さきほどのwの評価により、二度目のcountの評価結果は2となる。\\
\subsection{実行例}
{\small
\begin{verbatim}
;;; L-Eval input:
count
;;; L-Eval value:
1

;;; L-Eval input:
w
;;; L-Eval value:
10

;;; L-Eval input:
count
;;; L-Eval value:
2

\end{verbatim}
}

\section{Ex. 4.33}
\subsection{考え方}
l-eval.scmを編集して、text-of-quotation関数を呼び出す際にenvを引数として渡すようにしたほか、text-of-quotation関数を以下のように変更した。\\
{\small
\begin{verbatim}
(define (text-of-quotation exp env)
	(if (pair? (cadr exp))
		(eval (make-quotation (cadr exp)) env)
		(cadr exp)))

(define (make-quotation body)
	(if (null? body)
		'()
		(list 'cons (list 'quote (car body)) (make-quotation (cdr body)))))
\end{verbatim}
}
これにより、クォートしたリストも遅延リストを生じるようになった。
\subsection{実行例}
{\small
\begin{verbatim}
;;; L-Eval input:
(define (cons x y) (lambda (m) (m x y)))
;;; L-Eval value:
ok

;;; L-Eval input:
(define (car z) (z (lambda (p q) p)))
;;; L-Eval value:
ok

;;; L-Eval input:
(define (cdr z) (z (lambda (p q) q)))
;;; L-Eval value:
ok

;;; L-Eval input:
(define (length l) (if (null? l) 0 (+ 1 (length (cdr l)))))
;;; L-Eval value:
ok

;;; L-Eval input:
(length (cons (/ 1 0) (cons 3 '())))
;;; L-Eval value:
2

;;; L-Eval input:
(car (cons 3 (/1 0)))
;;; L-Eval value:
3

\end{verbatim}
}

\end{document}
